Are there different ways to categorize breast cancer based on protein expression data, with machine learning being used to classify them without using the pam50 proteins?
the data were used to assess how the mutations in the DNA are affecting the protein expression landscape in breast cancer.
Genes in our DNA are first transcribed into RNA molecules which then are translated into proteins.
Changing the information content of DNA has impact on the behavior of the proteome, which is the main functional unit of cells, taking care of cell division, DNA repair, enzymatic reactions and signaling etc.
In the analysis part for creating a Machine learning algorithem it was seen that the data dimensions are too big in the sens of attributes and too low in the sense of its instances.
This in turn created problems creating good models and in turn yielded very bad performing models.
The conclusion is that at this moment creating a good performing model takes alot more samples and research, and time for answering the question if it is possible to categorize breast cancer based on protein expression data, with machine learning being used to classify them without using the pam50 proteins?

